%% Literature %%
\documentclass[12pt]{article}
\usepackage{url}
\usepackage[a4paper, total={6in, 8in}]{geometry}
\usepackage{hyperref}
\usepackage{titlesec}

\setcounter{secnumdepth}{4}

\titleformat{\paragraph}
{\normalfont\normalsize\bfseries}{\theparagraph}{1em}{}
\titlespacing*{\paragraph}
{0pt}{3.25ex plus 1ex minus .2ex}{1.5ex plus .2ex}

\begin{document}

\title{WikiGaze}
\maketitle

\cite{wiki:Wikipedia:Statistics}. It has been consistently ranked in the top ten visited sites on the Internet as per Alexa.com. 

\cite{giles2005internet} This study shows that Wikipedia's content quality is comparable to traditional encyclopedias. 

\cite{bergstrom2009conversation} Search engines such as Google and Bing respond to user queries by extracting structured knowledge from Wikipedia.

\cite{cress2008systemic} Observes that collaborative knowledge building portals mimic the actual knowledge building process. 


\cite{antin2010readers} Even with "Anyone Can Edit" feature, it dominates in terms of information provision. On wikipedia, users play a variety of role where readers are also the part of knowledge building process.

\cite{rayner1998eye} Eye movement characteristic of reading is extensively studied in the psychology literature. Eye movement tracking open the door for an automated analysis of document reading. 

In~\cite{beymer2005webgazeanalyzer} a system is proposed for recording and analyzing web reading behavior using eye gaze.

Sanches et. al~\cite{sanches2018estimation} show that prediction of the subjective understanding is improved by 13\% if we use eye gaze instead of comprehension questions.

\cite{okoli2012people} performed on 477 Wikipedia based researches revealed that 42\% of these reseachers were centered on understanding characteristics of contributors involment or article quality evaluation~\cite{wilkinson2007assessing, kittur2008harnessing, stvilia2008information}. Only 20\% of the studies related to readers in Wikipedia and the usage of Wikipedia. Less than 1\% of the reviewed studies looked at users' reading preferences.


Lehmann et al.~\cite{lehmann2014reader} emphasized on the importance of reading behavior analyses. The research characterizes users' reading preferences at article level i.e. it determines which articles are more ``engaging" than others. 


In past studies~\cite{calder2002reading, bff9c00ddce3404ca729f4a96d53a701}, it was revealed that our eye gaze pattern is closely related to our thought process. Through eye gaze tracking, we can get the knowledge about the portions of an article where a user is more focused while reading. 


There is evidence in research from reading psychology that eye movement patterns while reading are indeed related to textual features~\cite{rayner1978eye}.


Personalization has been identified as being one of the grand challenges in information retrieval lately~\cite{belkin2008some}. 

Personalized summarization~\cite{berkovsky2008aspect} presents users with document extracts that are of interest to them. 

Majority of the automatic personalized summarization techniques incorporate some kind of personal information for individually improving the quality of summary~\cite{moro2012personalized, wu2008personalized, kumar2008generating}.

Well established devices, like Tobii~\cite{olsen2012tobii}, EyeLink~\cite{cornelissen2002eyelink}, etc. are available for gaze tracking.

CVC ET~\cite{ferhat2014cheap, CVC} is the port of Opengazer for the Linux and Mac platform and NetGazer~\cite{WinNT} is the port of Opengazer for the Windows platform.

Xu et al.~\cite{xu2009user} talk about the relevance between the human text reading pattern and their current cognitive process. Their basic assumption is that the eyes' fixation duration on a word is directly equivalent to the user's interest in that word.


Article length is considered an effective metric for analysing an article~\cite{chevalier2010wikipediaviz, dang2016quality}.   

Falagas et. al~\cite{falagas2013impact} demonstrates the impact of article length on the number of citations that it might get in future. 

Blumerstock ~\cite{blumenstock2008size} shows that article length is a critical parameter  to differentiate featured articles from other articles in Wikipedia. 


But it can not be said that all long articles are good.~\cite{blumenstock2008size} shows several counterexamples that prevent using it alone as a quality measure.



personalized document summarization:
1.) WebInEssence: A Personalized Web-Based Multi-Document
Summarization and Recommendation System
2.) Personalized PageRank Based Multi-document Summarization
3.) Automatic query-based personalized summarization that uses pseudo relevance feedback with NMF
4.) Personalized Document Summarization Using Non-negative Semantic Feature and Non-negative Semantic Variable
5.) Personalized text summarization based on important terms identification
6.) Summarize what you are interested in: An optimization framework for interactive personalized summarization


\bibliography{literature} 
\bibliographystyle{ieeetr}

\end{document}
