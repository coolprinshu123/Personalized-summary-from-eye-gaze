%%% VISION %%%%

\documentclass[12pt]{article}
\usepackage{url}
\usepackage[a4paper, total={6in, 8in}]{geometry}
\usepackage{hyperref}
\usepackage{titlesec}

\setcounter{secnumdepth}{4}

\titleformat{\paragraph}
{\normalfont\normalsize\bfseries}{\theparagraph}{1em}{}
\titlespacing*{\paragraph}
{0pt}{3.25ex plus 1ex minus .2ex}{1.5ex plus .2ex}

\begin{document}

\title{Vision (WikiGaze)}
\maketitle

The success of Wikipedia and the relative high quality of its
articles seem to contradict conventional wisdom. Recent
studies have begun shedding light on the processes
contributing to Wikipedia’s success, highlighting the role of editors and moderators. Very few studies focus on the importance of other end of spectrum i.e readers of Wikipedia articles. \cite{Beymer:2005:WSC:1056808.1057055}
The users of Wikipedia can be divided into two sets; the production team members(editors, moderators) and the passive team members(readers). Most of the researches performed to analyze various characteristics of Wikipedia revolve around understanding the behavior of the production teams, i.e. editors, moderators and their collaboration dynamics. A study~\cite{okoli2012people} performed on 477 Wikipedia based researches revealed that 42\% of these reseachers were centered on understanding characteristics of contributors involment or article quality evaluation~\cite{wilkinson2007assessing, kittur2008harnessing, stvilia2008information}. Only 20\% of the studies related to readers in Wikipedia and the usage of Wikipedia. Less than 1\% of the reviewed studies looked at users' reading preferences.

For a particular article, multiple users generate their personalized summaries. We ask the users to share their summaries for analyzes purpose.  Due to privacy reasons, some users might not be willing to share their data. As an incentive mechanism we implement a special feature to generate ``recommended summary". To unlock this feature, a user must share their data with the central analyzes team. Each user is given a special ID to maintain anonymity. We develop a cross-platform standalone application to generate personalized summary and to recommend summaries based on user's past reads.



%%%%%%%%%%%%%%%%%%%%%%%%%%%%%%%%%%%%%%%%%%
\section{Relation between Article Readability and Summary}

Eye movement patterns characteristic of reading, extensively studied in the
psychology literature~\cite{rayner1998eye}, open the door for an automated
analysis of web browser reading. 
While readers intuitively may think that their eye gaze follows a continuous left-to-right motion, eye tracking studies show that eye motion
advances in discrete chunks across the page. 
A reader's eyes will actually stop, or fixate, on a set of characters for
about 250 ms. 
This fixation is followed by a saccade, an eye movement of about 10 characters to the right, where the eyes will stop at the next fixation. 
A regression, or backwards eye movement in the text, is a sign that the
reader is having difficulty understanding the material. 

A high value of regression can be an indication of low readability of the text. The summary is being created by observing the reading pattern of the article. The summary generated for an article with low readability will contain so many sentences with jumbled sentence order. This can indicate be a nice way of research.




%%%%%%%%%%%%%%%%%%%%%%%%%%%%%%%%%%%%%%%%%%%%%
\section{Relation between article quality and gaze pattern}

In this thread, I plan to explore if there exists any relation between how people read an article and how it develops. Whether the articles which are read in sequential manner (i.e. linearly from one to another section) have more chances to become a "higher quality" article. Here quality indicates one of the quality labels from stub to FA.

I also plan to explore if the articles which are viewed more number of times and which has more "reading proportions" i.e. users are reading a good amount of the article when they refer it; tend to rank up in the quality leader or not.



\bibliography{vision} 
\bibliographystyle{ieeetr}

\end{document}
